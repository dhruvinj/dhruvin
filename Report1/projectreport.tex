\documentclass[11pt,a4paper]{article}
\usepackage[utf8]{inputenc}
\usepackage{amsmath}
\usepackage{amsfonts}
\usepackage{amssymb}
\usepackage{graphicx}
\usepackage{lmodern}
\usepackage{fourier}
\usepackage[left=2cm,right=2cm,top=2cm,bottom=2cm]{geometry}
\begin{document}
\title{Finite Element analysis of laminar flames}
\author{Dhruvin Naik}
\date{\today}
\maketitle
\begin{abstract}
\end{abstract}
\section{Governing Equations}
\subsection{Problem definition}

The flow of a compressible fluid is described in terms of the velocity $(u)$, pressure $(p)$, density $(\rho)$, and temperature$(T)$ fields. These fields are solutions of the compressible Navier Stokes equations that describe the dynamics of the system
and that are statements of conservation of mass, momentum and energy. The classical compressible navier stokes equation is written as \\
\begin{eqnarray}
\frac{D \rho^*}{D t^*}\hspace{1mm}+\hspace{1mm}\rho^*\nabla\cdot u^* &=& 0 \\
\rho^*\frac{D u^*}{D t^*}\hspace{1mm} &=&\hspace{1mm}-\nabla p^*+\hspace{1mm}\nabla\cdot\tau^*\hspace{1mm}+\hspace{1mm}\rho^* g^* \\
\rho^* c_p^* \frac{D T^*}{D t^*}\hspace{1mm}-\hspace{1mm}\beta^*T^* \frac{D p^*}{D t^*}\hspace{1mm}&=& \hspace{1mm} \tau^*\cdot\nabla u^*  + \hspace{1mm}\nabla\cdot q^* \\
\end{eqnarray}

Where $\tau^*$ is the viscous stresses. $g^*$ is the gravity vector. $c_p^*$ is the specific heat at constant pressure, $\beta^*$ is the thermal expansion coefficient , $q^*$ is the heat flux vector. For a newtonian fluid \\
$\tau^* = \mu^*(\nabla u^* +(\nabla u^*)^t)) - \frac{2}{3}\mu^*\nabla. u^* I$ \\
Assuming Fourier's law $q=-k \nabla T$.\\
$\mu^*$ is the viscocity and $k$ is the thermal conductivity.
\section{Introduction}
\section{Governing Equations}
\subsection{Problem definition}
The The flow of a compressible fluid is described in terms of the velocity $(u)$, pressure $(p)$, density $(\rho)$, and temperature$(\vartheta)$ fields. These fields are solutions of the compressible Navier Stokes equations that describe the dynamics of the system
and that are statements of conservation of mass, momentum and energy and a state equation relating the thermodynamic variables. The nondimensional form of energy equation is written as follows with the help of nondimensional numbers like strouhal number $S$, Mach number $M$, Reynolds number $R$, Prandtl number $Pr$, Froude number $F$, Heat release number $H$ and Temperature variation number $\varepsilon$. \\
$S=\frac{l_0}{u_0t_0}$,  $M=\frac{u_0}{\sqrt{p_0/\rho_0}}$, $R=\frac{\rho_ol_0u_0}{\mu_0}$, $P=\frac{\rho_0l_0u_0c_{p_0}}{k_0}$, \\ $F=\frac{u_0}{\sqrt{g_0l_0}}$, $H=\frac{Q_0t_0}{\rho_0\vartheta_0c_{p_0}}$, $\varepsilon=\frac{{\Delta}\vartheta}{\vartheta_0}$\\
where $l_0$, $t_0$, $\rho_0$, $p_0$, $\vartheta_0$, $u_0$, $\mu_0$, $k_0$, ${c_p}_0$, $g_0$, $Q_0$ and $\Delta\vartheta$ are the scales of length,
time, density, pressure, temperature, velocity, viscosity, conductivity, constant
pressure specific heat, external acceleration and external heat and temperature
variation respectively ($p_0,$ $\rho_0$ and$ \vartheta_0$ are assumed to be related by a state
equation).\\
The system of equation that needs to be solved reads

\begin{eqnarray}
S\frac{\partial \rho}{\partial t}\hspace{1mm}+\hspace{1mm}\nabla.\hspace{0.5mm}(\rho u) &=& 0 \\
\rho \left( S \frac{\partial u}{\partial t}\hspace{1mm}+\hspace{1mm}u.\nabla u \right)\hspace{1mm}+\hspace{1mm}\frac{1}{M^2} \nabla p\hspace{1mm}-\hspace{1mm}\frac{1}{R}\nabla. (2\mu\varepsilon'(u)) &=& -\frac{1}{F^2}\rho\hat{z} \\
\rho c_p \left( S \frac{\partial \vartheta}{\partial t}\hspace{1mm}+\hspace{1mm}u.\nabla \vartheta \right)\hspace{1mm}\hspace{1mm}-\hspace{1mm}\Gamma\beta\vartheta \left( S \frac{\partial p}{\partial t}\hspace{1mm}+\hspace{1mm}u.\nabla p \right)\hspace{1mm}-\hspace{1mm}\frac{M^2}{R}\Phi\hspace{1mm}-\hspace{1mm}\frac{1}{P}\nabla.(k\nabla\vartheta) &=& HSQ
\end{eqnarray}


$\hat{z}=(0,0,1)^t$, $\varepsilon'(u)=\varepsilon-\frac{1}{3}(\nabla.u)I$ where $\varepsilon=\frac{1}{2}(\nabla u+\nabla u^t)$ and $\beta$ is thermal expansion coefficient. $\Phi$ is rayleigh dissipation function defined as $\Phi=2\mu\varepsilon'(u):\varepsilon'(u)$ and $\Gamma=\frac{p_o}{\rho_0\vartheta_0c_{p_0}}$ which depends on the state equation. In case of an ideal gas $p=\rho\vartheta$ and 

$\Gamma=\frac{\gamma-1}{\gamma}$

The boundary conditions for momentum equation are
\begin{eqnarray}
u\hspace{1mm}&=& \hspace{1mm}u_D \\
\left(-\frac{1}{M^2}pI\hspace{1mm}+\hspace{1mm}\frac{1}{R}2\mu\varepsilon'(u)\right).n\hspace{1mm}&=&\hspace{1mm}\frac{1}{M^2}t \\
\end{eqnarray}
The boundary condition for energy equation is
\begin{eqnarray}
\vartheta\hspace{1mm}&=&\hspace{1mm}1\hspace{1mm}+\hspace{1mm}\varepsilon\vartheta_D \\
\frac{1}{p}kn.\nabla\vartheta\hspace{1mm}&=& \hspace{1mm}HSq \\
\end{eqnarray} \\
Finally the initial condition is

 $\xi(x,0)=\xi_0 (x)$

for $\xi=u,\hspace{1mm} \xi=p,\hspace{1mm} \xi=\rho, \hspace{1mm} \xi=\vartheta$ 
\subsection{Asymptotic Analysis}

The limit when the Mach number tends to zero can be found using standard
procedures of asymptotic analysis described. The first step
is to expand all flow variables in power series of the small parameter considered 

$\xi(x,t,M)=\xi^{(0)}(x,t)+M\xi^{(1)}(x,t)+M^2\xi^{(2)}(x,t)+ O(M^3)$ 


\noindent we will include a long space variable $\eta=Mz $ in the z direction to include the
case of slow atmospheric motion. Therefore, we propose the
expansion



$\xi(x,t,M)=\xi^{(0)}(x,\eta,t)+M\xi^{(1)}(x,\eta,t)+M^2\xi^{(2)}(x,\eta,t)+ O(M^3)$ 

In a bounded domain, we have

$\left. \frac{\partial \xi}{\partial z} \right|_M = \frac{\partial \xi^{(0)}}{\partial z} +\left(\frac{\partial \xi^{(0)}}{\partial\eta}+\frac{\partial\xi^{(1)}}{\partial z}\right)M+ +\left(\frac{\partial \xi^{(1)}}{\partial\eta}+\frac{\partial\xi^{(2)}}{\partial z}\right)M^2+O(M^3)$\\
Any physical property $\chi$ where $\chi$ can be $k$,$\mu$,$c_p$ and $\beta$ can be expanded as 

$\chi(\vartheta,p)=\chi^{(0)}+M\chi^{(1)}+M^2\chi^{(2)}+ O(M^3)$

The second step is to substitute asymptotic expansion into equations 1 to 3 and
to require that all terms in the expanded equations that are multiplied by
the same power of $M$ vanish to obtain a hierarchy of equations. When the
Mach number tends to zero and the rest of the numbers remain $O(1)$, keeping
the first set of equations of the hierarchy, we obtain the low Mach number
approximation discussed in subsection 2.3. The same procedure is followed
when other numbers (apart from the Mach number) tend to zero at the same
time but different results are obtained depending on the relation between them.
\subsection{Low Mach number equations}
After expanding all the variables by asymptotic expansion and equating similar order terms together, the following equations for mass conservation, momentum conservation and energy conservation respectively  are obtained 

\begin{eqnarray}
S\frac{\partial \rho^{(0)}}{\partial t}\hspace{1mm}+\hspace{1mm}\nabla.\hspace{0.5mm}(\rho^{(0)} u^{(0)}) &=& O(M^0) \\
M^{-2}\nabla p^{(0)} &=& O(M^{-2})\\
M^{-1}\left(\frac{\partial p^{(0)}}{\partial \eta}\hat{z}+\nabla p^{(1)}\right) &=& O(M^{-1})\\
\rho^{(0)} \left( S \frac{\partial u^{(0)}}{\partial t}\hspace{1mm}+\hspace{1mm}u^{(0)}.\nabla u^{(0)} \right)\hspace{1mm}+\hspace{1mm}\frac{\partial p^{(1)}}{\partial \eta}\hat{z}+\nabla p^{(2)}\hspace{1mm}-\hspace{1mm}\frac{1}{R}\nabla. (2\mu\varepsilon'(u^{(0)})) &=& O(M^0) \\
\rho^{(0)} c_p^{(0)} \left( S \frac{\partial \vartheta^{(0)}}{\partial t}\hspace{1mm}+\hspace{1mm}u^{(0)}.\nabla \vartheta^{(0)} \right)\hspace{1mm}\hspace{1mm}-\hspace{1mm}\Gamma^{(0)}\beta^{(0)}\vartheta^{(0)} \left( S \frac{\partial p^{(0)}}{\partial t}\hspace{1mm}+\hspace{1mm}u^{(0)}.\nabla p^{(0)} \right)\hspace{1mm}-\hspace{1mm}\frac{1}{P}\nabla.(k^{(0)}\nabla\vartheta^{(0)}) &=& HSQ \\
\end{eqnarray}

The asymptotic expansion of state equation gives following set of equation:
\begin{eqnarray}
p^{(0)}&=&\rho^{(0)}\vartheta^{(0)} \\
p^{(1)}&=&\rho^{(0)}\vartheta^{(1)}+\vartheta^{(0)}\rho^{(1)} \\
\end{eqnarray}

\subsection{Low mach number approximation}
This case is defined by $M \rightarrow0 $,$F = O(1)$, $H = O(1)$ and $\varepsilon= O(1)$. Then
the external forces are of $O(1)$ and from 11 we have $p^{(0)} = p^{(0)}(\eta, t)$ whereas
from 12 we have $p^{(1)} = p^{(1)}(\eta, t)$. In a bounded domain we have $p^{(0)} = p^{(0)}(t)$
and $p^{(1)} = p^{(1)}(t)$ which becomes irrelevant and can be taken constant. The pressure
splits into two contributions: $p^{(0)}$, a reference thermodynamic pressure and
$p^{(2)}$, a mechanical pressure. The first one, constant over the whole domain,
changes its value only by global heating or mass adding as shown below. The
mechanical pressure component $p^{(2)}$ is determined from a velocity constraint
playing the same role as in incompressible flows. In the zero Mach number
limit a system of equations for $\rho^{(0)}$, $\vartheta^{(0)}$, $p^{(2)}$ and $u^{(0)}$ has to be solved.
The reference pressure $p^{(0)}$, also called thermodynamic pressure, depends
on the boundary conditions of the problem. The thermodynamic
pressure is determined by the boundary condition. This can be seen introducing
the asymptotic expansion in the boundary condition, from where
\begin{eqnarray}
p^{(0)}&=&t^{(0)}.n \\
p^{(1)}&=&t^{(1)}.n \\
\left(-p^{(2)}I\hspace{1mm}+\hspace{1mm}\frac{1}{R}2\mu\varepsilon'(u^{(0)})\right).n\hspace{1mm}&=&\hspace{1mm}t^{(2)}
\end{eqnarray}

Using the zero order mass and energy conservation equations
and the state equation an equation relating the velocity divergence and the
thermodynamic pressure can be found. In the case of an ideal gas, this constraint
is
\begin{eqnarray}
p^{(0)}\nabla. u^{(0)} &=& -S\frac{1}{\gamma}\frac{dp^{(0)}}{dt}+ \frac{1}{P}\nabla.(k^{(0)}\nabla\vartheta^{(0)}) + HSQ
\end{eqnarray}
\section{second method}
\subsection{Euler and Navier stokes equations}
The conservation laws of mass, momentum and energy for an inviscid and viscous flow are
called the Euler and Navier-Stokes equations, respectively, in computational fluid dynamics.
The Navier-Stokes equations in differential conservative form read as follows:
continuity, momentum and energy equations are follows respectively:
\begin{eqnarray}
\frac{\partial \rho^*}{\partial t^*}\hspace{1mm}+\hspace{1mm}\nabla.\hspace{0.5mm}(\rho^* u^*) &=& 0 \\
\frac{\partial\rho^* u^*}{\partial t^*}\hspace{1mm}+\hspace{1mm}\nabla. (\rho^* u^* u^*) \hspace{1mm}+\hspace{1mm}\nabla p^* &=& G^* \\
\frac{\partial\rho^* E^*}{\partial t^*}\hspace{1mm}+\hspace{1mm}\nabla. (\rho^* H^* u^*) \hspace{1mm}&=& Q^* \\
\end{eqnarray}
$G^*$ is the sum of external forces. $Q^*$ represents the sum of work done by external forces and of heat release by external sources per unit volume per unit time. $E^*$ is the total energy which is sum of internal energy and kinetic energy. $H^*$ is enthalpy.
\begin{eqnarray}
G^*&=& \nabla.\tau^* + \rho^* g^* \\
\tau^* &=& \mu^*(\nabla u^* +(\nabla u^*)^t)) - \frac{2}{3}\mu^*\nabla. u^* I \\
H^* &=& E^* + \frac{p^*}{\rho^*} \\
E^* &=& e^* + \frac{1}{2}|{u^*}^2| \\
Q^* &=& \nabla.(\tau^* u^*) + \rho^* g^*.u^*+ \nabla.(k^* \nabla T^*)+ \rho^* q^* \\
p^* &=& \rho^* R^* T^* \\
e^* &=& {c_v}^* T^* \\
\gamma &=& \frac{c_p^*}{c_v^*}
\end{eqnarray}

\subsection{LOW MACH NUMBER ASYMPTOTICS}
We nondimensionalize the Equations by using reference quantities denoted by the subscript $\infty$,
e.g. farfield or stagnation conditions, and a typical length scale $L^*$ of the considered
flow. The thermodynamic reference quantities are assumed to be related by the equation of
state for a perfect gas. We define the nondimensional quantities by:\\
$\rho = \frac{\rho^*}{\rho_\infty} $, $p = \frac{p^*}{p_\infty} $, $u = \frac{u^*}{u_\infty} $, $T = \frac{T^*}{T_\infty} $, $\mu = \frac{\mu^*}{\mu_\infty} $, $k = \frac{k^*}{k_\infty} $,$x = \frac{x^*}{L^*} $, $t = \frac{t^*}{L^*/u^*_{\infty}} $, $e = \frac{e^*}{p^*_{\infty}/\rho^*_{\infty}} $, $E = \frac{E^*}{p^*_{\infty}/\rho^*_{\infty}} $, $H = \frac{H^*}{p^*_{\infty}/\rho^*_{\infty}} $\\
The Mach number is given by:
$$ M = \frac{u^*_{\infty}}{\sqrt{p^*_{\infty}/\rho^*_{\infty}}}= \sqrt{\gamma}M_\infty$$

Using the relations above. we may write the nondimensional
Navier-Stokes equations and other equations of interest as follows:
\begin{eqnarray}
\frac{\partial \rho}{\partial t}\hspace{1mm}+\hspace{1mm}\nabla.\hspace{0.5mm}(\rho u) &=& 0 \\
\frac{\partial\rho u}{\partial t}\hspace{1mm}+\hspace{1mm}\nabla. (\rho u u) \hspace{1mm}+\hspace{1mm}\frac{1}{M^2}\nabla p &=& G \\
\frac{\partial\rho E}{\partial t}\hspace{1mm}+\hspace{1mm}\nabla. (\rho H u) \hspace{1mm}&=& Q \\
E &=& e + M^2\frac{1}{2}|u^2| \\
p &=& \rho T \\
e &=& \frac{1}{\gamma-1} T \\
P &=& (\gamma-1)\left[\rho E- M^2\frac{1}{2} \frac{|\rho u|^2}{\rho}\right)
\end{eqnarray}

\noindent Where $Q =  \frac{M^2}{Re_\infty}\nabla.(\tau u) + \frac{M^2}{{Fr}^2_\infty} \rho g.u + \rho q + \frac{\gamma}{(\gamma-1) Re_\infty Pr_\infty} \nabla. (k\nabla T)$ \\
\subsection{Asymptotic analysis}

Each flow variable is expanded as 

$$p(x,t)=p_0(x,t)+Mp_1(x,t)+M^2p_2(x,t)+o(M^3)$$ \\
The zero order continuity equation is written as 
\begin{eqnarray}
\frac{\partial \rho^0}{\partial t^0}\hspace{1mm}+\hspace{1mm}\nabla.\hspace{0.5mm}(\rho^0 u^0) &=& 0
\end{eqnarray}

The zero, first and second order momentum equation are written as follows respectively
\begin{eqnarray}
M^{-2}\nabla p^{(0)} &=& 0\\
M^{-1}\nabla p^{(1)} &=& 0\\
\frac{\partial\rho_0 u_0}{\partial t}\hspace{1mm}+\hspace{1mm}\nabla. (\rho_0 u_0 u_0) \hspace{1mm}+\hspace{1mm}\nabla p^{(2)} &=& G_0 
\end{eqnarray}

The zero order energy equation is given by
\begin{eqnarray}
\frac{\partial\rho_0 E_0}{\partial t}\hspace{1mm}+\hspace{1mm}\nabla. (\rho_0 H_0 u_0) \hspace{1mm}&=& Q_0
\end{eqnarray}
In the above equations \\
$G_0 = \frac{1}{Re_\infty} \nabla.\tau_0 + \frac{1}{Fr^2_\infty}\rho_0 g$ and
$Q_0 = \frac{\gamma}{(\gamma-1) Re_\infty Pr_\infty} \nabla. (k\nabla T)_0 + (\rho q)_0$

It is assumed that $Re_\infty$ and $Pr_\infty$ are of order 1 i.e $M^0$. Since the work done by the viscous and buoyancy forces is of order $O(1)$. The zeroth- and first-order energy-source terms $Q_0$ are governed by heat-conduction and heat-release rate only, provided the Prandtl number $Pr$ is of order $O(1)$ and the Froude number $Fr^2$ is of order $O(Re_\infty Pr_\infty)$, provided that the ratio $\frac{\gamma}{\gamma-1}$ is of order $O(1)$. However, if the Reynolds number
is of the order $O(2)$ then the work done by
the viscous or buoyancy forces, respectively, will also contribute to the zeroth-order energy
source term $Q_0$

The asymptotic expansion of the state equation yields
\begin{eqnarray}
p_0 &=& (\gamma-1)(\rho E)_0 \\
p_1 &=& (\gamma-1)(\rho E)_1 \\
p_2 &=& (\gamma-1)(\rho E)_2 - \frac{1}{2}\rho_0 u^2_0  \\
T_0 &=& \frac{P_0}{\rho_0}
\end{eqnarray}

With $\rho_0 H_0 = (\rho E)_0 + p_0 = \frac{\gamma}{\gamma-1}p_0$ the zero order energy equation becomes 
\begin{eqnarray}
(\nabla. u)_0 &=& \frac{\gamma - 1}{(\gamma) p_0} Q_0 -\frac{1}{\gamma} \frac{dp_0}{dt} \\
(\nabla. u)_0 &=& \frac{\gamma - 1}{(\gamma) p_0} \left(\frac{\gamma}{(\gamma-1) Re_\infty Pr_\infty} \nabla. (k\nabla T)_0 + (\rho q)_0 \right) -\frac{1}{\gamma} \frac{dp_0}{dt}\\
(\nabla. u)_0 &=& \frac{\gamma - 1}{(\gamma) p_0} \left( \rho_0 {c_p}_0\left(\frac{\partial T_0}{\partial t} + u.\nabla T_0 \right)\right) -\frac{1}{\gamma} \frac{dp_0}{dt}
\end{eqnarray}











\end{document}
